%############################################################
\clearpage\section{To-Do-Notes}
Um bei einer längeren Arbeit nicht den Überblick zu verlieren, an welcher Stelle es nötig ist
weiter zu arbeiten, bietet es sich an, kleine Notizen einzufügen. Das Package \textit{todonotes}
stellt eine elegante Lösung bereit, um differenziert und vielfarbig jene Abschnitte zu kennzeichnen,
die einer weiteren Bearbeitung bedürfen.

\subsection*{Beispiel für To-Do-Notes}
Dies hier ist ein Blindtext zum Testen von Textausgaben. Wer diesen Text liest, ist selbst
\todo{Plain todonotes.} schuld. Der Text gibt lediglich den Grauwert der Schrift an. Ist das wirklich so? Ist es
gleichgültig, ob ich schreibe: Dies ist ein Blindtext? oder Huardest gefburn? Kjift?
mitnichten! Ein Blindtext bietet mir wichtige Informationen. An ihm messe ich die Les-
barkeit einer Schrift, ihre Anmutung, \todo{Plain todonotes.}wie harmonisch die Figuren zueinander stehen
und prüfe, wie breit oder schmal sie läuft. Ein Blindtext sollte möglichst viele verschie-
dene Buchstaben enthalten und in der Originalsprache gesetzt sein. Er muss keinen
Sinn ergeben, sollte aber lesbar sein.\todo[nolist]{Todonote that is only shown in the margin and not in
the list of todos.}%
Fremdsprachige Texte wie Lorem ipsum dienen
nicht dem eigentlichen Zweck, da sie eine falsche Anmutung vermitteln. Dies hier ist
ein Blindtext zum Testen von Textausgaben. Wer diesen Text liest, ist selbst schuld. 

\todo[inline]{A very long todonote that certainly will fill more
than a single line in the list of todos. Just to make sure let's add
some more text \ldots}

Der Text gibt lediglich den Grauwert der Schrift an. Ist das wirklich so? Ist es gleichgültig,
ob ich schreibe: Dies ist ein Blindtext? oder Huardest gefburn? Kjift? mitnichten!
\todo[noline]{A note with no line back to the text.}
Ein Blindtext bietet mir wichtige Informationen. An ihm messe ich die Lesbarkeit einer
Schrift, ihre Anmutung, wie harmonisch die Figuren zueinander stehen und prüfe, wie
\todo[caption={A short entry in the list of todos}]{A very long
todonote that certainly will fill more than a single line in the
list of todos \ldots}
breit oder schmal sie läuft. Ein Blindtext sollte möglichst viele verschiedene Buchstaben
enthalten und in der Originalsprache gesetzt sein. Er muss keinen Sinn ergeben, sollte
aber lesbar sein. Fremdsprachige Texte wie Lorem ipsum dienen nicht dem eigentlichen Zweck, 
da sie eine falsche Anmutung vermitteln.
\missingfigure{A figure I have to make \ldots}

%Nummerierte ToDo-Notes
\todox{Erste Nummer...}
\todox{Zweite Nummer...}

%Alles To-Dos als Liste ausgegeben
Nachfolgend wird noch eine Liste aller To-Dos auf einer separaten 
Seite ausgegeben.
%\begingroup
	%\let\clearpage\relax
	%\let\cleardoublepage\relax
	\listoftodos
%\endgroup 