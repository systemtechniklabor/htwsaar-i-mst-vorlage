\section{Abbildungen}
%=============================

\textit{LaTeX} unterstützt generell die Formate \textit{*.jpeg}, \textit{*.png} und \textit{*.pdf}.
Handelt es sich z.B. um Strichgrafiken oder skalierbare Farbflächen, sollte \textit{*.pdf} die erste Wahl sein,
da sich in diesem Format Vektorgrafiken ohne Qualitätsverlust darstellen bzw. skalieren lassen.

\begin{figure}[bth] 
  \centering
  \includegraphics[width=0.7\textwidth]{Examples/example_5.png}
  \caption{Erstes Bild, Völklinger Hütte}
  \label{fig:Huette}
\end{figure}


\subsection{Wrapfigure}
Abbildung~\ref{fig:Huette} ist zwar ganz nett anzusehen, aber vielleicht sähe es eleganter aus, wenn die Abbildung 
von unserem Textabschnitt umflossen werden würde. Diese Art von Abbildungen sollte jedoch sparsam und mit großer Sorgfalt eingesetzt werden, da es zu unschönen Darstellungen kommen kann.
\blindtext
\begin{wrapfigure}{l}{0.4\textwidth}
  \centering
  \includegraphics[width=0.4\textwidth]{Examples/example_5.jpg}
  \caption{Völklinger Hütte, *.jpg}
  \label{fig:Huette2}
\end{wrapfigure}
\blindtext


\subsection{Subfigures}
Es ist ebenso möglich, mehrere Abbildungen nebeneinander zu setzen, wie in Abbildung~\ref{fig:Beide} zu sehen ist. Eine separate Referenzierung ist auch möglich: Abbildung~\ref{subfig:abbone}.
\begin{figure}[bth]
  \subfloat[Erstes ...]{\includegraphics[width=0.49\textwidth]{Examples/example_5.png}\label{subfig:abbone}}\hfill
  \subfloat[... und zweites Bild]{\includegraphics[width=0.49\textwidth]{Examples/example_5.png}\label{subfig:abbtwo}}
  \caption{Mehrere Abbildungen nebeneinander}
  \label{fig:Beide}
\end{figure}


\subsection{Qualitätsunterschiede}
\begin{figure}[p]
	\centering
  \subfloat[\textit{PDF}-Format]{\includegraphics[width=0.65\textwidth]{Examples/bpmn.pdf}} \\
  \subfloat[\textit{JPG}-Format]{\includegraphics[width=0.65\textwidth]{Examples/bpmn.jpg}}
  \caption{Beide Formate im Vergleich}
  \label{fig:pdfvsjpg}
\end{figure}

\begin{figure}[p]
	\centering
  \includegraphics[width=0.65\textwidth]{Examples/jabref.png}
   \caption{\textit{PNG}-Format}
  \label{fig:pngvsjpg1}
\end{figure}

\begin{figure}[p]
	\centering
  \includegraphics[width=0.65\textwidth]{Examples/jabref.jpg}
   \caption{\textit{JPG}-Format}
  \label{fig:pngvsjpg2}
\end{figure}

Leider haben die unterschiedlichen Grafikformate bedingt durch die unterschiedlichen Kompressionsverfahren einige Schwächen, insbesondere die Umwandlung in das \textit{JPG}-Format erzeugt unangenehme Artefakte im Bild. \autoref{fig:pdfvsjpg} zeigt die Unterschiede zwischen \textit{PDF-Format} und \textit{JPG-Format} im Vergleich. 

Wenn eine \textit{*.pdf}-Datei nicht infrage kommt, beispielsweise bei Screenshots, ist unbedingt das \textit{PNG-Format} vorzuziehen. 
Den Unterschied machen \autoref{fig:pngvsjpg1} und \autoref{fig:pngvsjpg2} deutlich.

\glqq Faustregeln\grqq im Umgang mit Abbildungen:
\begin{itemize}
	\item Diagramme bzw. alles, was Linien usw. enthält: \textit{PDF} (im Vektorformat).
	\item Screenshots bzw. alles, was größere gleichfarbige Flächen enthält: \textit{PNG}.
	\item Der Rest (in der Regel Fotos): \textit{JPEG}.
\end{itemize}




